\documentstyle[rfc]{article}
%\rfc          - used when generating RFC instead of I-D
\rfcnum{98329138}
\i-d
\title{Authentication service over TCP}
\author{Arthur Poulet}
\address{University of Kent}
\pubdate{August, 2017}

\begin{document}

\maketitle

\begin{abstract}
	stuff
\end{abstract}

\newpage
\tableofcontents

\newpage
\section{Introduction}
\paragraph{}
In lots of projects it is required to have a authentication system, in order to authorize users to do some actions, and identify them for logging purposes. This is a critical security issue because an error in it cost money, trust, and privacy. Furthermore, it is very interesting for some parties to have a centralized authentication system shared for several services.
\paragraph{}
The goal of the Auth service is to provide a clear, secure, and simple authentication interface and authorization manager. It is a TCP server that SHOULD provide a SSL connection, on which an user must be able to authenticate, and request access to a resource.
\paragraph{}
The Auth service does not intend to replace security check on the services that use it. It is an annular similar to the LDAP services, but using a much simpler query language.

\newpage
\section{Terminology}
\begin{description}
	\item[client]
	\item[server]
	\item[auth-api]
	\item[secured client]
	\item[command]
	\item[response]
\end{description}

\newpage
\section{Protocol Overview}
\paragraph{}

\end{document}
